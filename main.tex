\documentclass[twocolumn]{aastex62}

\turnoffedit

\usepackage{amsmath}
\usepackage[flushleft]{threeparttable} 
\usepackage{array}
\setlength\extrarowheight{2pt}

\newcommand{\nTOri}{10} 
\newcommand{\nVbig}{39}
\newcommand{\nVsmall}{40}
\newcommand{\nLP}{11}
\newcommand{\nrobeam}{$21\arcsec$}
\newcommand{\resolutionpc}{$0.04$~pc}
\newcommand{\distance}{$414$~pc}
\newcommand{\Msun}{$M_\odot$}
\newcommand{\Mdot}{$M_\odot$~yr$^{-1}$}
\newcommand{\kms}{km~s$^{-1}$}

\newcommand{\co}[1][]{\ensuremath{^{#1}}CO}
\newcommand{\bothco}{\co[12] and \co[13]}

\let\vec\mathbf

\shorttitle{Outflows in Orion A}
\shortauthors{Feddersen {\em et al.}}

\begin{document}

\title{CARMA-NRO Orion: Protostellar Outflows in the Orion A Molecular Cloud}

\author{Jesse R. Feddersen}
\affiliation{Department of Astronomy, Yale University, P.O. Box 208101, New Haven, CT 06520-8101, USA}
\author{H\'ector G. Arce}
\affiliation{Department of Astronomy, Yale University, P.O. Box 208101, New Haven, CT 06520-8101, USA}
\author{Shuo Kong}
\affiliation{Department of Astronomy, Yale University, P.O. Box 208101, New Haven, CT 06520-8101, USA}

\email{jesse.feddersen@yale.edu}

\begin{abstract}

\end{abstract}

\keywords{ISM: clouds --- ISM: individual objects (Orion A) ---  stars: formation}

\section{Introduction}



\section{Observations}

\subsection{CO Maps}

\subsection{Source Catalogs}

\subsection{Other Outflow Studies}


\section{Description of Outflows}

\subsection{Outflow Identification}

\subsection{Driving Sources}


\section{Physical Properties of Outflows - draw from Dunham et al. 2014, Zhang et al. 2016, Tanabe et al. 2019}
To calculate the physical properties of outflows, we adapt methods described by \citet{Arce01}, \citet{Dunham14}, and \citet{ZhangY16}.

\subsection{Mass, Momentum, and Kinetic Energy}
To measure the outflow mass, we must first extract each outflow from the surrounding emission. 

\subsubsection{Excitation Temperature}

\subsubsection{$^{12}$CO Opacity Correction}
In Orion, \co[12] is usually optically thick \citep{Kong18}. 

\subsubsection{Low-Velocity Outflow Emission}
Outflows are conspicuous because of their high-velocity emission. But outflows also exist at lower velocities. \citet{Dunham14} note that escape velocities from protostars can be as low as 0.1 \kms. As these speeds are much lower than typical cloud CO line-widths. this low-velocity outflow material is often difficult to disentangle from the host molecular cloud.

The low-velocity contribution to the total outflow mass can be quite significant. \citet{Dunham14} found that adding the inferred low-velocity emission increased the mass of outflows by a factor of 7.7 on average, with some outflows increasing by an order of magnitude or more. Outflow momentum and energy increased by factors of 3 to 5, on average.  \citet{Offner11} found that high-velocity outflow emission underestimated the total outflow mass by a factor of 5 in synthetic observations of simulated outflows. Clearly, low-velocity emission should be accounted for when assessing the absolute impact of outflows on the cloud.

\subsection{Significance of Outflow Feedback }



\section{Outflow - Filament Alignment}
Outflows are ejected along the angular momentum axis of the protostellar disk. If mass accretion proceeds hierarchically, from larger scales of the cloud down to the protostellar scale, then the angular momentum axis of the protostar will trace the orientation of mass accretion flows. Protostars tend to form along narrow filaments of dense gas, and may accrete mass either along (parallel to) or across (perpendicular to) their host filaments. (CITE MOLECULAR FILAMENT PAPERS). If protostar predominately accrete mass along filaments, their outflows should be ejected perpendicular to the long axis of the filament. Conversely, if protostars accrete mostly across filaments, their outflows will be ejected parallel to the filament axis. 


In simulations, \citet{Li18} showed that moderately strong magnetic fields produce a perpendicular alignment between outflows and filaments. 

The alignment between CO outflows  and filaments alignment has been studied in Perseus by \citet{Stephens17} and in a massive infrared dark cloud (IRDC, \citealp{Kong19}). In Perseus, \citet{Stephens17} showed that outflows are consistent with being randomly oriented with respect to the filament, neither parallel nor perpendicular. In the IRDC G28.37+0.07, \cite{Kong19} showed that outflows are preferentially perpendicular to the filament axis. It remains to be seen whether this discrepancy arises from some meaningful difference between these clouds (e.g. evolutionary state or magnetic field strength) or by chance. 

In Orion, \citet{Davis09} showed that H$_{2}$ outflows appear randomly oriented on the sky, showing no preferential alignment to the North-South integral-shaped filament. \citet{Tanabe19} studied the CO outflows in single-dish CO maps and found no evidence for alignment between outflows and the large-scale filaments in the cloud. However, the filamentary structure in Orion A is more complex than a single North-South integral-shaped filament. We use the C$^{18}$O filaments identified by \citet{Suri19} in our analysis.

\subsection{$C^{18}O$ Filaments}





\section{Conclusions}

\bibliographystyle{aasjournal}
\bibliography{all.bib}
\end{document}